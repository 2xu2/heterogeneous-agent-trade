\documentclass[pdftex,11pt]{article}
\usepackage[pdftex]{graphicx,color}
\usepackage{setspace}
\usepackage{amsmath,amssymb}
\usepackage{titlesec}
\usepackage{subfigure}
\usepackage{fancyhdr}
\usepackage[longnamesfirst]{natbib}
\usepackage{cite}
\usepackage[paperwidth=8.5in,
left=0.5in,right=0.5in,paperheight=11.0in,textheight=9.5in,centering]{geometry}

\bibliographystyle{ecta}
\definecolor{nblue}{RGB}{0,0,128}

\usepackage[colorlinks=true, linkcolor=nblue,
citecolor=black, urlcolor=nblue, bookmarks=false,
pdfstartview={XYZ null null 0.70},
pdftitle={Redistributing the Gains From Trade Through Progressive Taxation},
pdfauthor={Spencer G. Lyon, Michael E. Waugh},
pdfkeywords={economics, trade, dynamics, quant econ, lyon, waugh, incomplete markets, taxes, redistribution, progressivity, inequality, Ricardo, julia, migration, taxation, social insurance}
]{hyperref}

\usepackage{setspace}

\onehalfspace

\renewcommand{\baselinestretch}{1.1}
\renewcommand{\arraystretch}{.7}
\setlength{\parindent}{0em}
\setlength{\parskip}{.1in}
\renewcommand\familydefault{\sfdefault}

\titleformat{\section}{\large\bf}{\thesection.}{.5em}{}
\titleformat{\subsection}{\normalsize\bf}{\thesubsection.}{.5em}{}
\titleformat{\subsubsection}{\normalsize\bf}{\thesubsubsection.}{.5em}{}

\def\thesection{\arabic{section}}
\def\thesubsection{\Alph{subsection}}
\def\thesubsubsection{\Roman{subsubsection}}

\newtheorem{proposition}{Proposition}
\newtheorem{assumption}{Assumption}

\pagestyle{fancy}
\rhead{}
\lhead{}
\cfoot{\thepage}
\lfoot{}
\lfoot{Revised: \today}
\renewcommand{\headrulewidth}{0pt}


\begin{document}

\subsection{Product Space}\label{sec:trade}

There are $M$ products. A firm has the following production technology to produce variety $j$:
\begin{align}
q_j = A_j N_j,
\label{eq:production}
\end{align}
where $N_j$ are the efficiency units of labor supplied by households to produce product $j$.

\subsection{Households}

There is a mass of $L_i$ households in each location $i$. Households are immobile across countries. They are infinite lived and have time-sparable preferences over non-durable consumption varieties:
\begin{align}
\small
E_{0} \sum_{t = 0}^{\infty} \beta^{t} u( \{ c_{jt} \}_{M}),
\end{align}
where the notation $\{ c_{jt} \}_{M}$ means that the household has preferences over all $j$ varieties. Households each period receive additive Type 1 extreme value shocks $\epsilon(j)$ with dispersion parameter $\sigma_{\epsilon}$ and then households make a discrete choice about which variety $j$ to each period and a continuous choice about how much. More specifically, the utility associated with the choice of variety $j$ is
\begin{align}
u( c_{jt} ) =  \frac{ c_{jt} ^{1-\gamma}}{1- \gamma} + \epsilon(j)_t. \label{eq:utility}
\end{align}
where consumption mapped into utils with a standard CRRA function, $\epsilon(j)$ is the taste shock.

A household's efficiency units are stochastic and they evolve according to a discrete state Markov chain. Mathematically, $z$ is a households efficiently units and $\mathcal{P}(z,z')$ describes the probability of a household with state $z$ efficiency units transiting to state $z'$.

Households can save and borrow in a non-state contingent asset $a$. The units of the asset are chosen to be the numeraire and pays out with gross interest rate $R$. I discuss this more in depth below, but the determination of $R_{i}$ is either exogenously given or the rate that clears the bond market. An country specific, exogenous debt limit $\phi$ constrains borrowing so:
\begin{align}
a_{t+1} \geq - \phi.
\label{eq:borrowing-constraint}
\end{align}
All these pieces come together in the household's budget constraint, conditional on choosing variety $j$ to consume, and focusing on a stationary setting where prices and transfers are constant:
\begin{align}
a_{t+1} + p_{ij}c_{jt}  \leq    R_{i} a_{t} + w_{i} z_{t} + \Pi_{i,\tau}.\label{eq:trade-budget-constraint}
\end{align}
The value of asset purchases and consumption expenditures must be less than or equal to asset payments, labor earnings, and transfers arising from firm profits.



\end{document}  